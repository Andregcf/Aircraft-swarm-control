\documentclass{article}
\usepackage[utf8]{inputenc}

\title{Aircraft swarm control}
\author{Andre Garcia Cunha Filho, Felipe Machini, Yuri Almeida, Vitor Taha Sant'Ana}
\date{06 de abril de 2021}

\begin{document}

\maketitle

\newpage

\section{Palavras-chave}

\begin{itemize}

    \item[1] Unmanned Swarm formation;
    \item[2] Drone delivery system;
    \item[3] Speed Bird Aero;
    \item[4] Colaborative UAVs;
    \item[5] Ground mapping;
    \item[6] The SCIADRO project
    
\end{itemize}       

\section{Revisão Bibliográfica}

Within the context of the 4.0 Industry, the world of business is seen as a Volatile, Uncertain, Complex and Ambiguous environment, also called the VUCA world (citar Klaus Shwab, the fourth industrial revolution). Such a harsh environment to do business requires new technologies, multidisciplinary high capable personnel and above all, disruptive ideas. It is also in exponential expansion, where new industries are constantly emerging and vanishing and in such a chaotic scenario, the Unmanned Aerial Vehicles (UAVs) arises as a highly efficient tool attracting the attention of several industries and its sectors.

\section{Escopo do projeto}

\begin{itemize}
    
    \item[\textbf{André}] Contextualizar globalmente os UAVs e seu papel na economia;
    
    \item[\textbf{Taha}] UAVs que trabalham em colaboração (swarms);
    
    \item[\textbf{Machini}] Expor o cenário de UAVs no Brasi;
    
    \item[\textbf{André}] Explicar como é feito o mapeamento utilizando UAVs no Brasil;
    
    \item[\textbf{Taha}] Contextualizar o tema: As grandes dificuldades envolvidas no mapeamento das grandes lavouras existentes no Brasil com um único drone em um único dia. Citar também o mapeamento com o auxilio de satelites;
    
    \item[\textbf{Machini Yuri Felipe }] Metologia do trabalho: uso do swarm com múltiplos mapeamentos simultaneos de uma mesma área.
    
\end{itemize}

\section{Se você estivesse sozinho, o que você faria?}

\subsection{André}

    Power line inspection, monitoring of cultural heritage sites, environmental monitoring, fire and gas detection, as well as precision agriculture.
    
    Urban monitoring, such as ensuring that the aerial power lines are free of obstacles like growing trees that could harm the line.
    
    Environmental monitoring
    
    First response to natural disasters (Which disasters occurs often in Brazil so we could think of?)
    
    Social events monitoring
    
    Use of several scanner drones to create 3D image of large buildings, such as historical constructions. In the same context, the inspection of historical buildings and bridges could benefit from the use of UAVs swarms, performing the maintenance job much quicker and safer than using human beings.
    
    Multipath Reatime transport protocol: http://wnlab.isti.cnr.it/ncmprtp.
    
    UAV swarm for mapping large areas. Consider in the flight plan wind conditions. Download the mission planer.
    
    \subsection{Felipe Machini}
    
    \subsection{Vitor}
    
    \subsection{Yuri}

\section{Escopo do Projeto}

Desenvolver o projeto de aeronaves autonomas em que uma ou mais aeronaves terão a capacidade de seguir uma outra aeronave guia (principal)

\section{Divisão das atividades}

\section{Etapas do projeto}

\begin{itemize}
    \item[1] Identificação dos modelos dinâmicos das aeronaves envolvidas;
    \item[2] Proposta de estratégia de controle baseado na missão em questão;
    \item[3] Validação numérico-computacional;
    \item[4] Implementação em hardware;
    \item[5] Validação experimental.
\end{itemize}

\end{document}
